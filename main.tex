\documentclass{article}
\usepackage[utf8]{inputenc}

\setlength{\parindent}{1em}
\setlength{\parskip}{.5em}

\title{Elaboration I}
\author{Jeremy Amin}
\date{August 2019}

\begin{document}

\maketitle

\tableofcontents

\section{Task Outline}

\noindent
In Scoping II I identified two patterns I wish to make easier. There were:

\begin{itemize}
    \item Finding specific words and sentences within and across different papers in an efficient manner
    \item Collecting and organizing papers into a single bibliographic database in such a way that I can search for papers based on a theme.

\end{itemize}

\noindent
Depending on the software I can find, I may modify the second bullet point to make it a simpler goal for my PoC. I will also include a referencing goal as a part of my PoC.

\begin{itemize}
    \item Find software which will serve as a reference database and which will do the referencing work for me 
\end{itemize}


\noindent
I notice that I did not make the Algorithm part of Scoping II detailed enough. It seems I misunderstood the difference between it and the Decomposition section. In the Elaboration process section I rectify this mistake.


\section{Potential Software to solve my problems}

\subsection{Voyant} 

Voyant seems like a good program for word and theme searches. This will solve some of the problems I identified in Scoping.

\subsection{Zotero}
Zotero seems like a sufficient program for referencing and as a bibliographic database. Alternatives include EndNote and Mendeley.

\subsection{Excel}

I think Excel will work well in conjunction with Voyant and Zotero as a way to thematically organise papers and other resources.

\subsection{OverLeaf}

Document writing with more control over the formatting etc. compared to Word or Google Documents. Also enables version control in a seamless way.

\subsection{GitHub}

Version control website.

\section{Elaboration process}

\begin{itemize}
    \item 
\end{itemize}

\end{document}
